\section{Interactive sessions}

%
% The idea I am going with for this section is to introduce a few
% filters. Give the audience a suggested image to load, to experiment
% with the filters as they are being describe. Then pose a problem
% which uses the prior filters for them to solve.
%

\begin{frame}
\frametitle{True utility of SimpleITK}
\begin{itemize}
\item Interacting with data
\end{itemize}
\end{frame}

\subsection{Threshold-based Segmentation}

\begin{frame}[fragile]
\frametitle{Data To Interact With}
\lstpython
\begin{lstlisting}
# Read image, using ipython's tab auto-complete
image = sitk.ReadImage( ``~/SimpleITK-MICCAI-2011-Tutorial/iasem-cells.nrrd'' )

# Get familiar with the image
print image
...
sitk.Show( image )
\end{lstlisting}
\begin{itemize}
  \item ``Dual-Beam'' or Ion-Abrasion Scanning Electron Microscope
  \item Heavily pre-proccessed
  \item  X-Z cross-section of a 3D volume
\end{itemize}
\end{frame}

\begin{frame}{Image Masks or Binary Images}
Image masks are just SimpleITK Images
\begin{itemize}
  \item Follow some conventions
  \item Pixel type of \texttt{uint8\_t}
  \item 0-value is background, 1-value being the foreground
  \item Masks are used for output of thresholding, binary morphology, etc\dots
  \item The 1-value was choosen to each of computataion with operators
  \item If a mask needs to be directly shown, multiply by 255
\end{itemize}
\end{frame}



\begin{frame}[fragile]
\frametitle{Threshold-based Segmentation}

\begin{itemize}
  \item {\bf Threshold}\\
    \small
    $ Output(x_i) =
    \begin{cases} Input(x_i) &\text{if $Lower \leq x_i \leq Upper$;}  \\
      OutsideValue            &\text{otherwise.}
    \end{cases} $
    \normalsize
  \item {\bf BinaryThreshold}\\
    \small
    $ Output(x_i) =
    \begin{cases} InsideValue &\text{if $LowerThreshold \leq x_i \leq UpperThreshold$;}  \\
      OutsideValue            &\text{otherwise.}
    \end{cases} $
    \normalsize
  \item {\bf OtsuThreshold} - Automatic Threshold values based on minimizing intra-class variance.
  \item {\bf DoubleThreshold} - A morphology based filters. Uses two sets of thresholds.
\end{itemize}

\lstpython
\begin{lstlisting}
# quick visualizations of masked image
sitk.Show( image * mask )
sitk.Show( .5*image*~mask+image*mask )
\end{lstlisting}

\end{frame}


\begin{frame}{Advanced Geodesic Morphology}

\begin{itemize}
  \item {\bf BinaryOpeningByReconstruction} - Removes binary elements which are smaller than the structuring element.
  \item {\bf BinaryClosingByReconstruction} - Fills binary holes which are smaller then the structuring element.
  \item {\bf BinaryFillHole} - Fills all holes in image.
  \item {\bf BinaryGrindPeak} - Removes all binary elements not connected to boarder.
\end{itemize}

\end{frame}

\begin{frame}{Change Border Problem}

After registration a border of the average intensity was added. Change this border to 0.

\end{frame}

\begin{frame}{Change Border Solution}
To iPython (\texttt{Examples/BasicTutorial2/InteractiveTutorial/05-01-BoarderChange.py})
\end{frame}

\subsection{Numpy Interface}
\begin{frame}[fragile]
\frametitle{SimpleITK with Numpy}

Functions to interface SimpleITK with numpy:

\lstpython
\begin{lstlisting}
def GetArrayFromImage(image):
    """Get a numpy array from a SimpleITK Image."""

def GetImageFromArray( arr ):
    """Get a SimpleITK Image from a numpy array."""
\end{lstlisting}
\begin{itemize}
  \item Both of these methods do a deep copy of the image
  \item Ensures safety, bypassing error-prone memory issues
\end{itemize}

\end{frame}

\begin{frame}{Python Example}
To iPython (\texttt{Examples/BasicTutorial2/InteractiveTutorial/05-02-Numpy.py})
\end{frame}

%
% Begin Section For Hans
%

\subsection{Image Measurements}
\begin{frame}{Image Statistics}
\end{frame}

\begin{frame}{Label Statistics}
\end{frame}


\begin{frame}{Statistics Problem}
\end{frame}

\begin{frame}{Statistics Solution}
\end{frame}


%
% End Section For Hans
%

\subsection{Feature Detection}


\begin{frame}[fragile]
\frametitle{More Data To Interact With}
\lstpython
\begin{lstlisting}
# Read image, using ipython's tab auto-complete
image = sitk.ReadImage( ``~/SimpleITK-MICCAI-2011-Tutorial/iasem-mito.nrrd'' )

# Get familiar with the image
print image
...
sitk.Show( image )
\end{lstlisting}
\end{frame}

\begin{frame}[fragile]
\frametitle{Edge Detection} 

\begin{itemize}
  \item {\bf CannyEdgeDetection}
\begin{lstlisting}
    Image CannyEdgeDetection ( const Image& ,
      double inLowerThreshold = 0.0,
      double inUpperThreshold = 0.0,
      std::vector<double> inVariance = std::vector<double>(3, 0.0),
      std::vector<double> inMaximumError = std::vector<double>(3, 0.01) );
\end{lstlisting}
  \item {\bf SobelEdgeDetection }
\begin{lstlisting}
    Image SobelEdgeDetection ( const Image&  );
\end{lstlisting}
  \item {\bf ZeroCrossingBasedEdge }
\begin{lstlisting}
    Image ZeroCrossingBasedEdgeDetection ( const Image& ,
      double inVariance = 1,
      uint8_t inForegroundValue = 1u,
      uint8_t inBackgroundValue = 0u,
      double inMaximumError = 0.1 );
\end{lstlisting}
  \item {\bf GradientMagnitudeRecursiveGaussian }
\begin{lstlisting}
    Image GradientMagnitudeRecursiveGaussian ( const Image& ,
      double inSigma = 1.0,
      bool inNormalizeAcrossScale = false );
\end{lstlisting}
\end{itemize}

\end{frame}


\begin{frame}[fragile]
\frametitle{Image Derivatives}

\begin{itemize}
  \item {\bf Derivative}
\begin{lstlisting}
    Image Derivative ( const Image& ,
      unsigned int inDirection = 0u,
      unsigned int inOrder = 1u,
      bool inUseImageSpacing = true );
\end{lstlisting}
  \item {\bf RecursiveGaussian}
\begin{lstlisting}
    Image RecursiveGaussian ( const Image& ,
      double inSigma = 1.0,
      bool inNormalizeAcrossScale = false,
      itk::simple::RecursiveGaussianImageFilter::OrderEnumType inOrder = itk::simple::RecursiveGaussianImageFilter::ZeroOrder,
      unsigned int inDirection = 0u );
\end{lstlisting}
\end{itemize}

\end{frame}

\begin{frame}[fragile]
\frametitle{Zero Crossing}
  

\begin{itemize}
  \item {\bf ZeroCrossing }
\begin{lstlisting}
    Image ZeroCrossing ( const Image& ,
      uint8_t inForegroundValue = 1u,
      uint8_t inBackgroundValue = 0u );
\end{lstlisting}
\end{itemize}

\end{frame}


\begin{frame}{Ridge and Valley Detection Problem}

Where $g(x;t)$ is a Gaussian function:\\
\center
\begin{center}
$g(x;t) = \frac{ 1 }{ \sqrt{ 2 t \pi } } e^{ \left( - \frac{x^2}{ 2 t} \right) }$ \\
\end{center}
Let $L$ denote a scale-space representaton of an image $I$:\\
\center
\begin{center}
$L(x,y;t) = = g(x,y; t) \ast I(x,y)$\\
\end{center}
And then $L_x$ is the partial derivative of the scale-space
representation of $I$ with respect  to $x$. Derivatives can also be
taken in other, directions. If $v$ is a direction parallell to image
gradient, then the following  defines a ridge:\\
\center
\begin{center}
$L_{uv} = 0, L_{uu}^2-L^2_{vv} \ge 0$
\end{center}

Where these directional derivatives have the following properties:
\center
\begin{center}
$L_v^2L_{uu}=L_x^2L_{yy}-2L_xL_yL_{xy}+L_y^2L_{xx},$\\
$L_v^2L_{uv}=L_xL_y(L_{xx}-L_{yy}) - (L_x^2-L_y^2)L_{xy},$\\
$L_v^2L_{vv}=L_x^2L_{xx}+2L_xL_yL_{xy} +L_y^2L_{yy}$\\
\end{center}

This view of the Ridge Detection if taken from Tony Lindeberg's works.

\end{frame}

\begin{frame}{Ridge Detection Example}
To iPython (\texttt{Examples/BasicTutorial2/InteractiveTutorial/05-04-RidgeDetection.py})
\end{frame}




